\documentclass[11pt,preprint]{elsarticle}

\usepackage{lmodern}
%%%% My spacing
\usepackage{setspace}
\setstretch{1.2}
\DeclareMathSizes{12}{14}{10}{10}

% Wrap around which gives all figures included the [H] command, or places it "here". This can be tedious to code in Rmarkdown.
\usepackage{float}
\let\origfigure\figure
\let\endorigfigure\endfigure
\renewenvironment{figure}[1][2] {
    \expandafter\origfigure\expandafter[H]
} {
    \endorigfigure
}

\let\origtable\table
\let\endorigtable\endtable
\renewenvironment{table}[1][2] {
    \expandafter\origtable\expandafter[H]
} {
    \endorigtable
}


\usepackage{ifxetex,ifluatex}
\usepackage{fixltx2e} % provides \textsubscript
\ifnum 0\ifxetex 1\fi\ifluatex 1\fi=0 % if pdftex
  \usepackage[T1]{fontenc}
  \usepackage[utf8]{inputenc}
\else % if luatex or xelatex
  \ifxetex
    \usepackage{mathspec}
    \usepackage{xltxtra,xunicode}
  \else
    \usepackage{fontspec}
  \fi
  \defaultfontfeatures{Mapping=tex-text,Scale=MatchLowercase}
  \newcommand{\euro}{€}
\fi

\usepackage{amssymb, amsmath, amsthm, amsfonts}

\def\bibsection{\section*{References}} %%% Make "References" appear before bibliography


\usepackage[numbers]{natbib}

\usepackage{longtable}
\usepackage[margin=2.3cm,bottom=2cm,top=2.5cm, includefoot]{geometry}
\usepackage{fancyhdr}
\usepackage[bottom, hang, flushmargin]{footmisc}
\usepackage{graphicx}
\numberwithin{equation}{section}
\numberwithin{figure}{section}
\numberwithin{table}{section}
\setlength{\parindent}{0cm}
\setlength{\parskip}{1.3ex plus 0.5ex minus 0.3ex}
\usepackage{textcomp}
\renewcommand{\headrulewidth}{0.2pt}
\renewcommand{\footrulewidth}{0.3pt}

\usepackage{array}
\newcolumntype{x}[1]{>{\centering\arraybackslash\hspace{0pt}}p{#1}}

%%%%  Remove the "preprint submitted to" part. Don't worry about this either, it just looks better without it:
\makeatletter
\def\ps@pprintTitle{%
  \let\@oddhead\@empty
  \let\@evenhead\@empty
  \let\@oddfoot\@empty
  \let\@evenfoot\@oddfoot
}
\makeatother

 \def\tightlist{} % This allows for subbullets!

\usepackage{hyperref}
\hypersetup{breaklinks=true,
            bookmarks=true,
            colorlinks=true,
            citecolor=blue,
            urlcolor=blue,
            linkcolor=blue,
            pdfborder={0 0 0}}


% The following packages allow huxtable to work:
\usepackage{siunitx}
\usepackage{multirow}
\usepackage{hhline}
\usepackage{calc}
\usepackage{tabularx}
\usepackage{booktabs}
\usepackage{caption}


\newenvironment{columns}[1][]{}{}

\newenvironment{column}[1]{\begin{minipage}{#1}\ignorespaces}{%
\end{minipage}
\ifhmode\unskip\fi
\aftergroup\useignorespacesandallpars}

\def\useignorespacesandallpars#1\ignorespaces\fi{%
#1\fi\ignorespacesandallpars}

\makeatletter
\def\ignorespacesandallpars{%
  \@ifnextchar\par
    {\expandafter\ignorespacesandallpars\@gobble}%
    {}%
}
\makeatother


% definitions for citeproc citations
\NewDocumentCommand\citeproctext{}{}
\NewDocumentCommand\citeproc{mm}{%
\href{\#cite.\detokenize{#1}}{#2}\nocite{#1}}

\makeatletter
% allow citations to break across lines
\let\@cite@ofmt\@firstofone
% avoid brackets around text for \cite:
\def\@biblabel#1{}
\def\@cite#1#2{{#1\if@tempswa , #2\fi}}
\makeatother
\newlength{\cslhangindent}
\setlength{\cslhangindent}{1.5em}
\newlength{\csllabelwidth}
\setlength{\csllabelwidth}{3em}
\newenvironment{CSLReferences}[2] % #1 hanging-indent, #2 entry-spacing
{\begin{list}{}{%
	\setlength{\itemindent}{0pt}
	\setlength{\leftmargin}{0pt}
	\setlength{\parsep}{0pt}
	% turn on hanging indent if param 1 is 1
	\ifodd #1
	\setlength{\leftmargin}{\cslhangindent}
	\setlength{\itemindent}{-1\cslhangindent}
	\fi
	% set entry spacing
	\setlength{\itemsep}{#2\baselineskip}}}
{\end{list}}

\usepackage{calc}
\newcommand{\CSLBlock}[1]{\hfill\break\parbox[t]{\linewidth}{\strut\ignorespaces#1\strut}}
\newcommand{\CSLLeftMargin}[1]{\parbox[t]{\csllabelwidth}{\strut#1\strut}}
\newcommand{\CSLRightInline}[1]{\parbox[t]{\linewidth - \csllabelwidth}{\strut#1\strut}}
\newcommand{\CSLIndent}[1]{\hspace{\cslhangindent}#1}


\urlstyle{same}  % don't use monospace font for urls
\setlength{\parindent}{0pt}
\setlength{\parskip}{6pt plus 2pt minus 1pt}
\setlength{\emergencystretch}{3em}  % prevent overfull lines
\setcounter{secnumdepth}{5}

%%% Use protect on footnotes to avoid problems with footnotes in titles
\let\rmarkdownfootnote\footnote%
\def\footnote{\protect\rmarkdownfootnote}
\IfFileExists{upquote.sty}{\usepackage{upquote}}{}

%%% Include extra packages specified by user

%%% Hard setting column skips for reports - this ensures greater consistency and control over the length settings in the document.
%% page layout
%% paragraphs
\setlength{\baselineskip}{12pt plus 0pt minus 0pt}
\setlength{\parskip}{12pt plus 0pt minus 0pt}
\setlength{\parindent}{0pt plus 0pt minus 0pt}
%% floats
\setlength{\floatsep}{12pt plus 0 pt minus 0pt}
\setlength{\textfloatsep}{20pt plus 0pt minus 0pt}
\setlength{\intextsep}{14pt plus 0pt minus 0pt}
\setlength{\dbltextfloatsep}{20pt plus 0pt minus 0pt}
\setlength{\dblfloatsep}{14pt plus 0pt minus 0pt}
%% maths
\setlength{\abovedisplayskip}{12pt plus 0pt minus 0pt}
\setlength{\belowdisplayskip}{12pt plus 0pt minus 0pt}
%% lists
\setlength{\topsep}{10pt plus 0pt minus 0pt}
\setlength{\partopsep}{3pt plus 0pt minus 0pt}
\setlength{\itemsep}{5pt plus 0pt minus 0pt}
\setlength{\labelsep}{8mm plus 0mm minus 0mm}
\setlength{\parsep}{\the\parskip}
\setlength{\listparindent}{\the\parindent}
%% verbatim
\setlength{\fboxsep}{5pt plus 0pt minus 0pt}



\begin{document}



\begin{frontmatter}  %

\title{Netflix Rescue}

% Set to FALSE if wanting to remove title (for submission)




\author[Add1]{Tagishi Mashego}
\ead{}








\vspace{1cm}





\vspace{0.5cm}

\end{frontmatter}

\setcounter{footnote}{0}



%________________________
% Header and Footers
%%%%%%%%%%%%%%%%%%%%%%%%%%%%%%%%%
\pagestyle{fancy}
\chead{}
\rhead{}
\lfoot{}
\rfoot{\footnotesize Page \thepage}
\lhead{}
%\rfoot{\footnotesize Page \thepage } % "e.g. Page 2"
\cfoot{}

%\setlength\headheight{30pt}
%%%%%%%%%%%%%%%%%%%%%%%%%%%%%%%%%
%________________________

\headsep 35pt % So that header does not go over title




\section{\texorpdfstring{Introduction
\label{Introduction}}{Introduction }}\label{introduction}

I am helping another billion dollar company stay rich.

\subsection{Part 1}\label{part-1}

Netflix being an international company has to cater for the likes of a
wide range of people , to help make them alter the algorithm in certain
regions in a more effective wqy. I have identified the five
highest-rated genres by production country. The following illustration
presents this analysis:

\begin{figure}[H]

{\centering \includegraphics{Question3_files/figure-latex/Figure1-1} 

}

\caption{Top Genre by Production Country \label{Figure1}}\label{fig:Figure1}
\end{figure}

Figure \ref{Figure1} shows that production countries based in EG , KW
and GB all have `comedy, drama' as their top- rated genre. The average
IMDb score of these countries is just below 8 indicating that audiences
really approve of not only the genre mix but the execution of these
production countries. Further , we see that the top rated genre for
Indian producers is documentaries , and for Spain and Argentina it
drama. However , it is evident that the overwhelming favorite is the
`comedy , drama' mix.

With an inexhaustible list of movies available , Netflix aims to select
a mix of movies and shows that the audiences will respond well to. A key
consideration in this process is whether viewers are more responsive to
older or newer titles. The analysis below explores this question.

\begin{figure}[H]

{\centering \includegraphics{Question3_files/figure-latex/Figure2-1} 

}

\caption{Trend of IMDb Scores \label{Figure2}}\label{fig:Figure2}
\end{figure}

Figure \ref{Figure2} shows that a decline in average IMDb scores movies
and shows released in the period between 1970 and 1980 , with the
average score picking up in the late nighties with some classics that
still make me laugh until the average IMDb scores peaked around 2010 and
since has been on the decline once more. Simply put , Netflix we need
more 90s classics. From a business standpoint, many individuals of
working age likely view the 1990s with a sense of nostalgia.
Capitalizing on this sentiment could be a key to higher retention.

\newpage

\subsection{Part 2}\label{part-2}

A select group of directors have gained popularity for their longevity
and consistent output in film and television production. Among them is
Marcus Raboy, best known for directing Friday After Next. As illustrated
in Figure \ref{Figure3}, he holds the highest number of titles available
on the Netflix platform. However, this raises an important question ,
does this superior performance compared to others in the top 10?

\begin{figure}[H]

{\centering \includegraphics{Question3_files/figure-latex/Figure3-1} 

}

\caption{Most Prolific Directors \label{Figure3}}\label{fig:Figure3}
\end{figure}

To address the question I have posed , the density plot below offers
valuable means of examining the distribution of IMDb scores and
clarifying any uncertainty.

\begin{figure}[H]

{\centering \includegraphics{Question3_files/figure-latex/Figure4-1} 

}

\caption{Density Plot of Top 5 Directors \label{Figure4}}\label{fig:Figure4}
\end{figure}

Figure \ref{Figure4} shows that the stand out director is Jay Karas who
on average produces the highest-rated content amongst the top 5 most
prolific directors. While Marcus Raboy's content spans a range, most of
his titles fall into the solidly average-to-good category. Notably, Jay
Chapman shows the highest level of consistency in the ratings of his
content, with a narrowly concentrated distribution centered around the
upper to mid range of the IMDb scale.

\hfill

\newpage

\section*{References}\label{references}
\addcontentsline{toc}{section}{References}

\phantomsection\label{refs}
\begin{CSLReferences}{1}{1}
\bibitem[\citeproctext]{ref-Texevier}
Katzke, N.F. 2017. \emph{{Texevier}: {P}ackage to create elsevier
templates for rmarkdown}. Stellenbosch, South Africa: Bureau for
Economic Research.

\end{CSLReferences}

\section*{Appendix}\label{appendix}
\addcontentsline{toc}{section}{Appendix}

\subsection*{Appendix A}\label{appendix-a}
\addcontentsline{toc}{subsection}{Appendix A}

Some appendix information here

\subsection*{Appendix B}\label{appendix-b}
\addcontentsline{toc}{subsection}{Appendix B}

Katzke (\citeproc{ref-Texevier}{2017})

\bibliography{Tex/ref}





\end{document}
